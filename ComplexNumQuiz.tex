%% Default Latex document template
%%
%%  blake@rcs.ee.washington.edu

\documentclass[letterpaper]{article}

% Uncomment for bibliog.
%\bibliographystyle{unsrt}

\usepackage{graphicx}
\usepackage{lineno}
%\usepackage{fancyhdr}

%%%%%%%%%%%%%%%%%%%%%%%%%%%%%%%%%%%%%%%%5
%
%  Set Up Margins
\input{templates/pagedim.tex}

%
%        Font selection
%
%\renewcommand{\rmdefault}{ptm}             % Times
%\renewcommand{\rmdefault}{phv}             % Helvetica
%\renewcommand{\rmdefault}{pcr}             % Courier
%\renewcommand{\rmdefault}{pbk}             % Bookman
%\renewcommand{\rmdefault}{pag}             % Avant Garde
%\renewcommand{\rmdefault}{ppl}             % Palatino
%\renewcommand{\rmdefault}{pch}             % Charter


%%%%%%%%%%%%%%%%%%%%%%%%%%%%%%%%%%%%%%%%%%%%%%%%%
%
%         Page format Mods HERE
%
%Mod's to page size for this document
\addtolength\textwidth{0cm}
\addtolength\oddsidemargin{0cm}
\addtolength\headsep{0cm}
\addtolength\textheight{0cm}
%\linespread{0.894}   % 0.894 = 6 lines per inch, 1 = "single",  1.6 = "double"

% header options for fancyhdr

%\pagestyle{fancy}
%\lhead{LEFT HEADER}
%\chead{CENTER HEADER}
%\rhead{RIGHT HEADER}
%\lfoot{Hannaford, U. of Washington}
%\rfoot{\today}
%\cfoot{\thepage}



% Make table rows deeper
%\renewcommand\arraystretch{2.0}% Vertical Row size, 1.0 is for standard spacing)

\begin{document}
\setpagewiselinenumbers        %  Line numbers for edits to drafts.
\modulolinenumbers[1]          %  number every N lines

% \linenumbers                   %  start numbering lines here

\clearpage
\newpage
%%%%%%%%%%%%%%%%%%%%%%%%%%%%%%%%%%%%%%%%%%%%%%%%%%%%%%%%%%%%%%%%%%%%%%%%%%%%%%%%%%%%%%%%%%%%%%%%%%%%%%%%%%%%%%%%%%%%%%
\section{Complex Number Quiz}\label{ComplexNumberQuiz}
Take this quiz then check your answers on Page \pageref{CN_answers}.  Use only the following functions on your calculator (or fewer as instructed):

\[
* \quad \div + \quad - \quad \sqrt{x}
\]


It should be {\bf easy for you to get exact answers}.  If not, then you need to review the concepts in this quiz and section \ref{cnconcepts}.   Some Kahn Academy videos are pre-linked in Section \ref{KahnV}.


\begin{enumerate}

\item  What is $\sqrt{-16}$ ?

\item  Evaluate
\[
X = \frac{-b + \sqrt{4ac}}{2a}
\]
for the following values:
\begin{quotation}
\begin{tabular} {c|c|c}
a&b&c  \\ \hline
1&2&3 \\
1&-4&29\\
2&28&1156
\end{tabular}
\end{quotation}


\item  Evaluate
\[
(6+j16) + (-7-j6) =
\]
\[
(27-j0.75) - (1.6+j0.27) =
\]


\item  Evaluate  $M\times N$ where

\begin{quotation}
\begin{tabular} {c|c}
M&N \\ \hline
$(2+6j)$	&	$(1+3j)$   	\\
$(1.7-0.6j)$    &	$(3.2+0.4j)$	\\
\end{tabular}
\end{quotation}


\item  Plot the following points on the complex plane:
\[
a = -3+1.5j \qquad b = 2-j \qquad c = j
\]


\includegraphics[width=6cm]{figsapdx/00926a.png}



\item  Convert $X_1=(4+3j)$ to polar (magnitude-angle) form


\item  Convert $X_2=(-16+3.7j)$ to polar (magnitude-angle) form


\item  Represent $X_3 = (-1+6j)$ in exponential form

\item  For
\[
a = 3e^{j\pi/4} \qquad b = 2\angle{45^\circ}
\]
Convert them to ``$a+bj$'' form and multiply $a*b$ without using a calculator.

\end{enumerate}





%  Use name of bibliography files without .bib extension
%\bibliography{brl}
\end{document}

